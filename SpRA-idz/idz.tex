\problemset{Специальные разделы алгебры}
\problemset{Индивидуальное домашнее задание}
\problemset{Вариант №41 (8123034)}

\renewcommand*{\proofname}{Решение}

\begin{problem}
	Функция $f : (\alpha;+\infty)\rightarrow(\beta;+\infty)$ задана формулой $f(x)=2x^2-7x+3$. Найдите наименьшие $\alpha$ и $\beta$, при которых функция $f$ биективна.
\end{problem}

\begin{proof}
	Функция биективна в той области, где она одновременно и инъективна и сюръективна.
	\newline
	Функция инъективна тогда, когда она строго монотонна. Поскольку верхний предел дан в условии, найдем минимальное значение на оси абсцисс, начиная с которого функция $f$ возрастает.
	\newline
	Построим график функции и отметим на нем вершину параболы красной точкой:
	\begin {center}
		\begin{tikzpicture}
			\begin{axis}[
				xmin=-10, xmax=11,
				ymin=-50, ymax=160,
				axis lines=middle,
				xlabel=$x$,
				ylabel=$y$
			]
				\addplot [color=gray, domain=-10:10] {2*x^2-7*x+3};
				\node[draw, color=red,label={[align=left]above:{\scalebox{0.5}{$1.75, -3.125$}}},circle,fill,inner sep=2pt] at (axis cs:1.75,-3.125) {};
			\end{axis}
		\end{tikzpicture}
	\end {center}
	Очевидно, что как раз от вершины функция начинает монотонно возрастать. Следовательно, $\alpha$ равна абсциссе вершины параболы.
	\newline
	Функция сюръективна тогда, когда ее область значения непрерывна. Очевидно, что область значения данной функции непрерывна начиная от вершины параболы и выше. Следовательно, $\beta$ равна ординате вершины параболы.
	\newline 
	Ответ: $\alpha=1.75, \beta=-3.125$.
\end{proof}

\begin{problem}
	Является ли функция $f:(1,\ldots,8)\rightarrow(1,\ldots, 6)$ заданная таблицей
	$\begin{pmatrix}
		1 & 2 & 3 & 4 & 5 & 6 & 7 & 8\\
		3 & 4 & 1 & 1 & 2 & 5 & 6 & 1
	\end{pmatrix}$ инъективной? сюръективной? биективной?
\end{problem}

\begin{proof}
	Функция является инъективной, если элементы ее образа не повторяются. Как видно из таблицы, $3$, $4$ и $8$ переходят в $1$. Следовательно, функция $f$ не инъективна.
	\newline
	Функция является сюръективной, если все элементы образа имеют соответствующий им элемент в прообразе. Так как образ содержит все числа от $1$ до $6$, функция $f$ сюръективна.
	\newline
	Функция биективна тогда, когда она и инъективна и сюръективна одновременно. Следовательно, $f$ не биективна.
\end{proof}

\begin{problem}
	\begin{minipage}[t]{\linegoal}
		\begin{enumerate}[leftmargin=*]
			\item Является ли группой множество корней 6-й степени из 1 с операцией сложения?
			\item Является ли группой множество невырожденных верхнетреугольных матриц размера $n\times n$ над $\mathbb {R}$ с операцией умножения?
			\item Является ли группой множество дробно-рациональных функций на расширенной вещественной оси с операцией композиции?
		\end{enumerate}
	\end{minipage}
\end{problem}

\begin{proof}
	Множество является группой относительно операции $\cdot$, если операция $\cdot$ ассоциативна, в множестве есть нейтральный элемент относительно $\cdot$, и для любого элемента множества существует обратный ему элемент.
	\newline
	Рассмотрим данные три случая:
	\begin{enumerate}
		\item Множество корней 6-й степени из 1 - подмножество множества $\mathbb {C}$. В множестве $\mathbb {C}$ операция сложения ассоциативна. Единственный нейтральный элемент множества $\mathbb {C}$ - $(0, 0)$ не принадлежит множеству корней 6-й степени из 1. Следовательно, в нём нет нейтрального элемента, и группой оно не является.
		\item Операция умножения любых матриц ассоциативна. Ее нейтральный элемент - единичная матрица - является элементом множества невырожденных верхнетреугольных матриц размера $n\times n$. Обратные матрицы (обратные элементы) для верхнетреугольных матриц также принадлежат этому множеству. Следовательно, это множество являктся группой.
		\item Множество дробно-рациональных функций не является замкнутым относительно операции композиции, а, следовательно, не является и группой.
	\end{enumerate}
\end{proof}

\begin{problem}
	Записать перестановку
	$\begin{pmatrix}
	1 & 2 & 3 & 4 & 5 & 6 & 7 & 8 & 9 & 10\\
	3 & 7 & 8 & 10 & 1 & 2 & 6 & 4 & 9 & 5
	\end{pmatrix}$ в виде произведения независимых циклов и найти ее порядок.
\end{problem}

\begin{proof}
	Обозначим перестановку квадратными скобками, а циклы - круглыми.
	Разложим на циклы: $[3\;7\;8\;10\;1\;2\;6\;4\;9\;5]=(1\;3\;8\;4\;10\;5)(2\;7\;6)(9)$
	\newline
	Порядок перестановки равен НОК длин всех простых циклов: $6$
\end{proof}

\begin{problem}
	Найдите произведение перестановок: $(1\;3\;7\;8\;4)(5\;6)\cdot(1\;5\;6\;7\;3\;8)(2\;4)$, ответ в циклической форме.
\end{problem}

\begin{proof}
	Представим произведения в виде перестановки:
	\[
		(1\;3\;7\;8\;4)(5\;6) = (3\;2\;7\;1\;6\;5\;8\;4)
	\]
	\[
		(1\;5\;6\;7\;3\;8)(2\;4) = (5\;4\;8\;2\;6\;7\;3\;1)
	\]
	Перемножим перестановки:
	\[
		(3\;2\;7\;1\;6\;5\;8\;4)\cdot(5\;4\;8\;2\;6\;7\;3\;1) = (1\;3\;7\;8\;4)(5\;6)\cdot(1\;5\;6\;7\;3\;8)(2\;4) = (6\;1\;4\;2\;5\;8\;7\;3)
	\]
	Представим перестановку в циклической форме:
	\[
		(6\;1\;4\;2\;5\;8\;7\;3) = (1\;6\;8\;3\;4\;2)(5)(7) = (1\;6\;8\;3\;4\;2)
	\]
\end{proof}

\begin{problem}
	Пусть $G = \mathbb {R} \ {-1}$. Зададим операцию $\star$ формулой $x\star y = x + y + xy$. Проверьте, что $(G,\star)$ - группа.
\end{problem}

\begin{proof}
	Множество является группой относительно операции $\star$, если операция $\star$ ассоциативна, в множестве есть нейтральный элемент относительно $\star$, и для любого элемента множества существует обратный ему элемент.
	\newline
	Так как $G$ - подмножество $\mathbb {R}$, а операция $\star$ состоит из операций сложения и умножения, которые, в свою очередь, ассоциативны на $\mathbb {R}$, операция $\star$ ассоциативна на $G$.
	\newline
	Нейтральный элемент для операции $\star$, определенной на $G$ - $0$, так как он является нейтральным для операции сложения, а при умножении обращает результат в $0$.
	\newline
	Обратный элемент для операции $\star$ вычисляется по формуле: $y=-\frac{x}{x+1}$. Исключение из множества $\mathbb {R}$ элемента $\{-1\}$ никак не повлияло на тот факт, что для каждого элемента $G$ можно найти обратный, так как $-1$ не имеет обратного элемента (уравнение $\frac{x}{x+1}-1=0$ не имеет решений).
	\newline
	Следовательно, множество $G$ - группа.
\end{proof}

\begin{problem}
	Пусть $\mu$ подгруппа в $\mathbb {C}$*, состоящая из всех корней степени 3 из 1. Докажите, что $\mathbb {C}$*$/\mu\cong\mathbb {C}$*.
\end{problem}

\begin{proof}
	Как известно, корни из 1 образуют группу по умножению:
	\newline
	Любая степень корня из 1 тоже является корнем из 1.
	\newline
	Обратный элемент для каждого из элементов группы совпадает с сопряжённым ему. В данном случае, $\frac{-1+i\sqrt{3}}{2}\cdot\frac{-1-i\sqrt{3}}{2}=1$.
	\newline
	Нейтральным элементом является комплексная единица.
	\newline
	Докажем, что $\mathbb {C}$*$/\mu\cong\mathbb {C}$*:
	\newline
	По критерию гомоморфизма групп для функции $\phi \in Hom(A, B)$:
	\[
		A/ker(\phi)\cong im(\phi)
	\]
	Пусть существует такая функция $\phi:\mathbb {C}\rightarrow\mathbb {C}$, что $ker(\phi)=H$, а $im(\phi)=\mathbb {C}$.
	\newline
	Зададим составную функцию $\phi$:
	\[
		\phi(x)=
		\begin{dcases}
			1,\;x\in H\\
			x-1,\;x\in H^+\\
			x,\;x\in\mathbb {C}/(H\cup H^+)
		\end{dcases}
	\]
	\[
		H^+=\{y+n|y\in H,n\in\mathbb {N}\}
	\]
	Очевидно, что $\phi$ удовлетворяет условиям. Следовательно, $\mathbb {C}$*$/ker(\phi)\cong im(\phi)\rightarrow\mathbb {C}$*$/H\cong\mathbb {C}$*
\end{proof}

\begin{problem}
	\begin{minipage}[t]{\linegoal}
		\begin{enumerate}[label=(\alph*), leftmargin=*]
			\item Гомоморфизм групп. Эпиморфизм, мономорфизм, изоморфизм. Ядро и образ гомоморфизма. Свойства ядра и образа.
			\item Чему равен образ гомоморфизма $\phi:G\rightarrow H$, если его ядро равно $G$?
		\end{enumerate}
	\end{minipage}
\end{problem}

\begin{proof}
	\begin{minipage}[t]{\linegoal}
		\begin{enumerate}[label=(\alph*), leftmargin=*]
			\item Гомоморфизм групп $(A,\cdot)$ и $(B,*)$ - это такое отображение $\phi:A\rightarrow B$, что $\forall x,y\in A:\phi(x\cdot y)=\phi(x)*\phi(y)$. Другими словами, гомоморфизм сохраняет алгебраическую структуру.
			\newline
			Эпиморфизм - это сюръективный гомоморфизм.
			\newline
			Мономорфизм - это инъективный гомоморфизм.
			\newline
			Изоморфизм - это биективный гомоморфизм.
			\newline
			Ядро гомоморфизма - это множество $ker(\phi)=\{x\in A|\phi(x)=e_B\}$
			\newline
			Образ гомоморфизма - это множество $im(\phi)=\{\phi(y)\in B|y\in A\}$
			\newline
			Свойства ядра и образа гомоморфизма:
			\begin{enumerate}
				\item $\phi(e_A) = e_B$
				\item $\forall x\in A:\phi(x^{-1})=(\phi(x))^{-1}$
				\item $ker(\phi)\trianglelefteq A$
				\item $im(\phi)\leq A$
				\item $A/ker(\phi)\cong im(\phi)$
			\end{enumerate}
			\item Если $ker(\phi)=G$, значит $\forall x\in G:\phi(x)=e_H$. Следовательно, $\{\phi(y)\in B|y\in A\}=\{e_H\}$. $im(\phi)=e_H$.
		\end{enumerate}
	\end{minipage}
\end{proof}
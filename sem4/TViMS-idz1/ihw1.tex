\problemset{Теория вероятностей и математическая статистика}
\problemset{Индивидуальное домашнее задание №1}
\problemset{Вариант №22}

\renewcommand*{\proofname}{Решение}

\begin{problem}
	Слово "математика"\ разрезается на буквы. Затем из этих букв случайным образом составляется слово. Определить вероятность, что получится то же слово.
\end{problem}

\begin{proof}
	Вероятность получения того же слова представляет из себя произведение вероятностей того, что на каждом шаге будет выбрана правильная буква из оставшихся. Необходимо учитывать, что некоторые буквы в слове одинаковы.
	\[	
		\frac{2}{10}\cdot\frac{3}{9}\cdot\frac{2}{8}\cdot\frac{1}{7}\cdot\frac{1}{6}\cdot\frac{2}{5}\cdot\frac{1}{4}\cdot\frac{1}{3}\cdot\frac{1}{2}\cdot1\approx0.0000066
	\]
	Ответ: $0.0000066$.
\end{proof}

\begin{problem}
	Из двух колод по 36 карт вытащили по 3 карты и переложили в 3 такую же колоду. После этого из 3 колоды вытащили 8 карт. Среди них оказалось 4 туза. Найти вероятность того, что переложили 2 туза.
\end{problem}

\begin{proof}
	Информация про 8 вытащенных из 3 колоды карт на решение задачи не влияет, так как 4 туза в 3 колоде могло быть при любом исходе первого действия.
	\newline
	Рассмотрим три варианта, при которых могло быть переложено 2 туза:
	\begin{enumerate} 
		\item Из 1 колоды был переложен 1 туз, и из второй колоды был переложен 1 туз
		\item Из 1 колоды было переложено 2 туза, а из второй не было переложено ни одного
		\item Из 1 колоды не было переложено ни одного туза, а из второй было переложено 2 туза
	\end{enumerate}
	Очевидно, что, поскольку 1 и 2 колоды одинаковые, вероятность 2 и 3 варианта равна.
	\newline
	Найдём вероятность того, что из 3 переложенных карт 2 были тузами:
	\[
		\frac{\frac{4!}{2!\cdot2!}\cdot\frac{32!}{31!\cdot1!}}{\frac{36!}{33!\cdot3!}}\approx0.02689
	\]
	Найдём вероятность того, что из 3 переложенных карт 1 была тузом:
	\[	
		\frac{\frac{4!}{3!\cdot1!}\cdot\frac{32!}{30!\cdot2!}}{\frac{36!}{33!\cdot3!}}\approx0.07721
	\]
	Найдём вероятность того, что из 3 переложенных карт ни одна не была тузом:
	\[	
		\frac{\frac{4!}{4!\cdot0!}\cdot\frac{32!}{29!\cdot3!}}{\frac{36!}{33!\cdot3!}}\approx0.69468
	\]
	Тогда для рассмотренных ранее вариантов:
	\begin{enumerate} 
		\item: $P(a)=0.07721^2=0.00596$
		\item = (3): $P(a)=0.02689\cdot0.69468=0.01868$
	\end{enumerate}
	Следовательно, общая вероятность будет равна сумме вероятностей рассмотренных двух случаев:
	\[
		P(a)=0.00596+0.01868\cdot2=0.04332
	\]
	Ответ: $0.04332$.
\end{proof}

\begin{problem}
	Система охраны некоторого объекта, имеющего форму прямоугольника $500\times300$м, состоит из трех контуров ограждений, соединяющихся между собой через каждые 100м. Вероятность повреждения каждого участка контура между двумя соседними соединениями независимо от других равно 0,2. Определить вероятность того, что данная система охраны будет нарушена.
\end{problem}

\begin{proof}
	Рассмотрим каждый участок ограждения, состоящий из трех контуров. Вероятность повреждения такого участка равно произведению составляющих его участков контуров:
	\[	
		0.2\cdot0.2\cdot0.2=0.008
	\]
	Общая вероятность будет равна сумме вероятностей повреждения всех рассмотренных участков:
	\[
		P(a)=0ю008\cdot((3+5)\cdot2)=0.128
	\]
	Ответ: $0.128$.
\end{proof}

\begin{problem}
	Вероятность успеха в схеме Бернулли равна $0.001$. Проводится $2000$ испытаний. Написать точную формулу и вычислить приближенно вероятность того, что число успехов не превышает $4$.
\end{problem}

\begin{proof}
	Согласно формуле Бернулли:
	\[	
		P_n(k)=\frac{n!}{k!\cdot(n-k)!}\cdot p^k\cdot(1-p)^{n-k}
	\]
	Вероятность того, что число успехов не превышает $4$ равна сумме вероятностей того, что число успехов будет равно $0$, $1$, $2$, $3$ и $4$ соответственно:
	\[
		P_n(k\leq4)=\sum_{i=0}^{4}\frac{n!}{i!\cdot(n-i)!}\cdot p^i\cdot(1-p)^{n-i}
	\]
	Поскольку $n\cdot p=2000\cdot0.001=2$ для приближения используем теорему Пуассона:
	\[
		P(k)\approx\frac{\lambda^k}{k!}\cdot\exp^{-\lambda}, где \lambda=n\cdot p=2
	\]
	\[
		P_n(k\leq4)\approx\sum_{i=0}^{4}(\frac{2^0}{0!}+\frac{2^1}{1!}+\frac{2^2}{2!}+\frac{2^3}{3!}+\frac{2^4}{4!})\cdot\exp^{-2}=(1+2+2+\frac{3}{4}+\frac{2}{3})\cdot\exp^{-2}=7\cdot\exp^{-2}\approx0.94735
	\]
	Ответ: $0.94735$.
\end{proof}